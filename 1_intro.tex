\label{intro}
Ce mémoire, à la jonction de l'histoire des sciences et de la linguistique computationnelle, propose une étude interdisciplinaire dont l'objectif est la valorisation numérique du fonds patrimonial de Jean-Martin Charcot, fondateur de la neurologie moderne au XIX\ieme{} siècle en France\footnote{Le présent travail fait également partie du projet doctoral de l'autrice en cours, dans le cadre du programme \textit{Instituts et Initiative} de l'Observatoire des Patrimoines de l'Alliance Sorbonne Université : \url{https://theses.fr/s382733}.}. À ce titre, nous nous intéressons tout particulièrement à l'analyse de la genèse et de la migration du discours médical de pathologie anatomique, de neurologie et psychologie de Charcot dans les écrits réalisés en collaboration et dans les écrits de ses disciples et continuateurs. Si l'importance de ses travaux scientifiques est un sujet largement étudié du point de vue théorique \citep{bogousslavsky2011following,broussolle2012,camargo2024}, 
%il n'existe pas, à notre connaissance, d'initiatives en humanités numériques visant l'exploitation\smalltodo{motivation} de ses archives. 
cet aspect reste inexploré dans une perspective quantitative.

Ce travail se veut donc un premier pas vers l'établissement de l'édition numérique du corpus Charcot issu de son fonds volumineux et d'une grande importance scientifique. D'abord, les traitements sous-jacents (notamment, la lemmatisation et l'indexation des textes par les lemmes) nous ont permis de produire une transcription interrogeable dans notre cadre de recherche grâce aux outils développés au sein de l'équipe-projet \textsc{ObTIC}\footnote{\url{https://obtic.sorbonne-universite.fr/}}. L'objectif de cette démarche est d'y analyser le discours médical de Charcot à travers l'extraction des expressions à mots multiples \citep[p. 96]{nerima2006}\footnote{angl. \textit{multi-word expressions}, se déclinant sous la forme suivante, entre autres : \textsc{substantif + adjectif + adjectif}. Exemple : la pathologie \textit{sclérose latérale amyotrophique}. 
%aujourd'hui connu sous le nom de \textit{maladie de Charcot} ou \textit{maladie Lou-Gehrig}.
}, qui constituent potentiellement des champs lexicaux et des savoirs en circulation. Ensuite, nous comparons les textes écrits par Charcot avec ceux de ses collaborateurs et successeurs, \textit{via} les concepts-clés liés à son discours scientifique. Nous considérons que ces concepts correspondent aux termes reflétant des contributions que Charcot a apportées à la compréhension et à la caractérisation des pathologies neurologiques (\textit{hystérie, sclérose en plaques disséminées}, etc.).
%\footnote{connue également sous le nom de \textit{sclérose multiple}.} etc.). 

Les expériences menées sur le transfert des concepts d'un corpus à l'autre se basent sur deux mesures statistiques et l'une de l'apprentissage profond\footnote{angl. \textit{deep learning}.} . Enfin, une proposition de l'extraction des phrases-clés à l'aide de l'apprentissage profond et de l'analyse sémantique des passages contenant les concepts médicaux est formulée. Au-delà du cas de Charcot, ce travail vise à établir un protocole permettant d'appréhender la circulation de concepts de manière
automatisée.

Le présent mémoire est structuré en quatre parties principales : après l'introduction, nous traçons l'évolution du progrès médical dans le cadre épistémologique, où Charcot a joué un rôle important, avant de présenter ses contributions principales (chapitre \ref{1_rupture}). Dans le chapitre \ref{sota}, nous soulignons les aspects des circulations des savoirs et proposons une revue de la littérature portant sur ce sujet du point de vue numérique. Le fonds Charcot et la constitution du corpus de recherche correspondant sont abordés dans le chapitre \ref{3_corpus}. Ensuite, le chapitre \ref{4_methodo} présente les premières tentatives d'analyse computationnelle de l'impact de Charcot sur ses élèves et collègues, ainsi que les limites de ces approches, en proposant de nouvelles méthodes pour la quantification de la pertinence des expressions polylexicales. Enfin, le dernier chapitre est consacré à la conclusion du travail et aux pistes pour des recherches futures.










