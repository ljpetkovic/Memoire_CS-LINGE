\label{intro}
Ce mémoire, à la jonction de l'histoire des sciences et de la linguistique computationnelle, propose une étude interdisciplinaire dont l'objectif est la valorisation numérique du fonds patrimonial de Jean-Martin Charcot, fondateur de la neurologie moderne au XIX\ieme{} siècle en France. Si l'importance de ses travaux scientifiques est un sujet largement étudié \citep{bogousslavsky2011following,broussolle2012,camargo2024}, il n'existe pas, à notre connaissance, d'initiatives en humanités numériques visant l'exploitation\smalltodo{motivation} de son fonds. Nous souhaitons donc mesurer informatiquement l'impact scientifique des travaux de Charcot sur ses collaborateurs et successeurs, membres de son réseau scientifique. Cette mesure se fonde sur l'analyse des concepts-clés en matière de son discours scientifique, et plus particulièrement sur l'opérationnalisation du terme \og{}influence\fg{}, définie ici comme une intertextualité uni-directionnelle, allant des écrits de Charcot (ci-après corpus \og{}Charcot\fg{}) vers ceux de ses collaborateurs et successeurs (ci-après corpus \og{}Autres\fg{}). Il s'agit donc \textit{in fine} d'aborder computationnellement la question des circulations, non pas des artefacts matériels comme les manuscrits \citep{gabay2021katabase} et les images \citep{joyeux2019visual}, mais des phénomènes textuels complexes \citep{manjavacas} ayant une dimension théorique forte.

Le présent mémoire est structuré en quatre parties principales \smalltodo{structure}: après l'introduction, nous traçons l'évolution du progrès médical dans le cadre épistémologique, où Charcot a joué un rôle important, avant de présenter ses contributions principales (chapitre \ref{1_rupture}). Dans le chapitre \ref{sota}, nous soulignons les aspects des circulations des savoirs et proposons une revue de la littérature portant sur ce sujet du point de vue numérique. Le fonds Charcot et la constitution du corpus de recherche correspondant sont abordés dans le chapitre \ref{3_corpus}. Ensuite, le chapitre \ref{4_methodo} présente les premières tentatives d'analyse computationnelle de l'impact de Charcot sur ses élèves et collègues, ainsi que les limites de ces approches, en proposant de nouvelles méthodes pour la quantification de la pertinence des expressions polylexicales. Enfin, le dernier chapitre est consacré à la conclusion du travail et aux pistes pour des recherches futures.










