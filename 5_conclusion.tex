

%Rappel du context
%Intro / Rappel Contexte

%Nous avons donc pu en tirer la problématique suivante :

Ce travail constitue la première phase d'exploration du corpus de Charcot. Les deux outils, \textsc{OBVIE} et \textsc{TextPair}, nous offrent des fonctionnalités avancées de recherche et de comparaison de textes dans le cadre d'une analyse de textes assistée par ordinateur. Cependant, ils ne proposent pas de fonctionnalité de lecture distante permettant de rendre compte de l'impact de Charcot sur son réseau scientifique à travers les concepts principaux de ses travaux. L'analyse effectuée à l'aide d'un nouveau script nous a alors permis de quantifier les concepts polylexicaux dans les deux corpus, selon trois différentes métriques de pondération. La visualisation des résultats nous a permis d'observer des phénomènes qu'il serait nécessaire de valider auprès de spécialistes de Charcot. Cette expérience répond partiellement à notre question de recherche, puisqu'elle ne comprend pas de dimension chronologique de l'impact des concepts médicaux et de leur étendue sur le long terme.

Pour la suite, deux pistes de recherche devraient être suivies : 1. 2. améliorer le texte issu de l'\textsc{OCR} à l'aide d'une approche basée sur l'apprentissage profond\footnote{Cette approche constituerait une suite du travail effectué dans le cadre de la correction automatique de l'\textsc{OCR} à l'aide du modèle de langue statistique issu du logiciel \texttt{JamSpell} \citep{petkovic2022impact}.} et évaluer l'impact de la correction orthographique de notre corpus sur ces résultats.

%Rappel des résultats


