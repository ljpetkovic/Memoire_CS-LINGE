

%Rappel du context
%Intro / Rappel Contexte

%Nous avons donc pu en tirer la problématique suivante :

Ce travail constitue la première phase d'exploration du corpus de Charcot qui vise à mesurer le degré d'intertextualité entre le discours médical de Charcot et celui de son réseau scientifique, en repérant les termes les plus importants de son \oe{}uvre dans les textes de ses collègues et successeurs. Les deux outils, \textsc{OBVIE} et \textsc{TextPair}, nous offrent des fonctionnalités avancées de recherche et de comparaison de textes dans le cadre d'une analyse de textes assistée par ordinateur ; or, ils ne proposent pas de fonctionnalité de lecture distante permettant de rendre compte de l'impact de Charcot sur son réseau scientifique à travers la pertinence des concepts principaux de ses travaux. Pour pallier ce problème, nous avons d'abord opté pour une approche statistique et supervisée, dans le cadre de laquelle nous avons quantifié les concepts polylexicaux dans les deux corpus selon trois différentes métriques de pondération : \textsc{TF-IDF}, \textsc{BM25} et \textsc{BERT}. Nous avons ensuite comparé cette approche avec celle basée sur l'apprentissage profond et non-supervisé, en extrayant les phrases-clés les plus pertinentes avec les outils \texttt{keybert} et \texttt{keyphrase-vectorizers}. Cela nous a permis de détecter les termes communs entre les deux corpus et enfin de montrer la répartition des termes les plus pertinents dans le réseau de Charcot. La visualisation des résultats nous a permis d'observer des phénomènes qu'il serait nécessaire de valider auprès de spécialistes de Charcot. Ces expériences répondent donc partiellement à notre question de recherche, puisqu'elles ne comprennent pas de dimension chronologique de l'impact des concepts médicaux sur le long terme.

Pour la suite, deux pistes de recherche devraient être suivies : dans un premier temps, opérer une analyse sémantique des passages avec l'outil Ariane qui contiendraient les concepts médicaux, afin d'étudier les différentes modalités de prise en charge énonciative : opinions, accords, désaccords, définitions, etc. \citep{alrahabi2021ariane}. En effet, reprendre un terme ne veut pas dire y adhérer : on peut citer pour dire, par exemple, que l'on n'est pas d'accord. 
Il serait donc pertinent d'annoter le corpus Charcot avec Ariane, retenir les passages qui contiennent les plus de catégories \og{}opposées\fg{} (puisque ses termes ont suscité du débat et de la polémique), y identifier les phrases-clés avec \texttt{keybert} ou \texttt{keyphrase-vectorizers}, et les comparer enfin avec les termes de l'index utilisé dans la partie \ref{methodo_stat}. Comme deuxième piste, nous proposons d'améliorer le texte issu de l'\textsc{OCR} à l'aide d'une approche basée sur l'apprentissage profond\footnote{Cette approche constituerait une suite du travail effectué dans le cadre de la correction automatique de l'\textsc{OCR} à l'aide du modèle de langue statistique issu du logiciel \texttt{JamSpell} \citep{petkovic2022impact}.} et évaluer l'impact de la correction orthographique de notre corpus sur ces résultats.

%Rappel des résultats


