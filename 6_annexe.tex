%Ne pas numéroter cette partie
\part*{Annexe}
%Rajouter la ligne "Annexes" dans le sommaire
%\addcontentsline{toc}{chapter}{Annexe}

\chapter*{Liste des termes et expressions popularisées par Charcot}



%changer le format des sections, subsections pour apparaittre sans le num de chapitre
\makeatletter
\renewcommand{\thesection}{\@arabic\c@section}
\makeatother

%recommencer la numérotation des section à "1"
\setcounter{section}{0}

%\section*{Liste des termes et expressions popularisées par Charcot}

\addcontentsline{toc}{section}{\protect\numberline{}Liste des termes et expressions popularisées par Charcot}%

\begin{multicols}{2}[\columnsep2.5em] 
\begin{table}[H]
\footnotesize
\begin{tabular}{l|l|l}
amblyopie hystérique & chorée & embolie\\
achromatopsie hystérique & chorée rythmique hystérique & encéphalite\\
amyotrophie protopathique & cicatrice vicieuse & endocardite\\
amyotrophie symptomatique & cirrhose de muscles & épilepsie\\
analgésie & c\oe{}lialgie & épilepsie spinale\\
anesthésie & compression de l'ovaire & éruption\\
angioneuroses & congestion & érythème pernios\\
apoplexie spinale & contractilité électrique & escarre des fesses\\
arthrite déformante & contracture hystérique permanente & escarre sacrée\\
arthropathie des ataxiques & contracture permanente & état de mal épileptique\\
articulations & contracture tardive & état de mal hystéro-épileptique\\
ataxie locomotrice & contracture des uretères  & excitabilité\\
atrophie musculaire & convulsionnaire & faisceau radiculaire interne \\
atrophie progressive & convulsion & faradisation\\
attaque-accès & corde du tympan & fève de calabar\\
attaque apoplectiforme & corps granuleux & \textit{glossy skin} \\
attaque hystérique & corps opto-strié & globe hystérique\\
attitude passionnelles & courant électrique & griffe\\
attraction & crise gastrique & hématomyélie\\
aura hystérique & danse & hémianesthésie hystérique\\
avant-mur & décubitus aigu & hémianesthésie de cause encéphalique\\
bromure de camphre & dégénération cireuse & hémichorée\\
bulbe rachidien & délire & hémiopie\\
capsule interne & diplopie & hémiparaplégie\\
capsule surrénale & dynamométrie &  hémiplégie\\
catalepsie & ecchymoses & histologie\\
cellule nerveuse & ecthyma & hypérémié\\
chloroforme & electro-diagnostic & hyperesthésie ovarienne\\
\end{tabular}
\end{table}
\columnbreak
\end{multicols}
\newpage

\begin{multicols}{2}[\columnsep2.5em] 
\begin{table}[H]
\footnotesize
\begin{tabular}{l|l|l}
hystérie & oblitération & vertige \\
hystérie épileptiforme & oligurie hystérique & vision \\
hystérie ovarienne & ovarie hystérique & vomissement hystérique\\
hystérie grave &  paralysie agitante & vomissement urémique\\
hystérie locale & paralysie bulbaire &  vomissement de sang\\
hystérie infantile & paralysie consécutive & zona\\
hystérie locale traumatique & paralysie générale progressive & \\
hystéro-épilepsie & paralysie générale spinale & \\
immobilisation & paralysie hystérique & \\
incoordination motrice & paralysie infantile & \\
ischémie & paralysie labio-glosso-laryngée & \\
ischurie & paralysie pseudo-hypertrophique & \\
latéropulsion & paralysie rhumatismale &  \\
lèpre & paraplégie traumatique & \\
lésion & parésie & \\
lésion oculaire & petit mal & \\
maladie de Parkinson & phlegmon & \\
méningite ascendante & pied bot & \\
méningite cervicale & putamen & \\
métalloscopie &  rémission &\\
miracle &  rétention & \\
moëlle épinière &  rétropulsion & \\
myélite aiguë centrale & rigidité & \\
myélite partielle & salivation &  \\
myélite traumatique & sclérodermie &  \\
myodynie & sclérose fasciculée & \\
myopathie & sclérose descendante & \\
néphro-cystite & sclérose latérale & \\
néphrotomie & sclérose postérieure & \\
nerf dilatateur & sclérose en plaque disséminées & \\
nerf facial & somnambulisme & \\
nerf glandulaire &  tarentisme & \\
nerf sciatique & torticolis & \\
nerf sécréteur &  thermoanesthésie & \\
nerf trijumeau & tremblement & \\
nerf trophique & trépidation & \\
nerf vaso-moteur & trismus & \\
névrite & trouble trophiques & \\
névroglie &  tubercule de la moëlle & \\
nitrite d'amyle & tympanisme & \\
nystagmus & urticaire & \\
\end{tabular}
\end{table}
\columnbreak
\end{multicols}

\begin{landscape}
\thispagestyle{mylandscape} 
%Call our predefined page type
% Please add the following required packages to your document preamble:
% \usepackage{multirow}
% Please add the following required packages to your document preamble:
% \usepackage{multirow}
% Please add the following required packages to your document preamble:
% \usepackage{multirow}
% Please add the following required packages to your document preamble:
% \usepackage{multirow}
% Please add the following required packages to your document preamble:
% \usepackage{multirow}
% \usepackage[table,xcdraw]{xcolor}
% If you use beamer only pass "xcolor=table" option, i.e. \documentclass[xcolor=table]{beamer}
% Please add the following required packages to your document preamble:
% \usepackage{multirow}
% \usepackage[table,xcdraw]{xcolor}
% If you use beamer only pass "xcolor=table" option, i.e. \documentclass[xcolor=table]{beamer}
\begin{table}[]
\centering
\begin{tabular}{|l|cccc|cccc|}
\hline
\multicolumn{1}{|c|}{}                                 & \multicolumn{4}{c|}{\cellcolor[HTML]{FFFC9E}\textit{Charcot}}                                        & \multicolumn{4}{c|}{\cellcolor[HTML]{CBCEFB}\textit{Autres}}                                         \\ \cline{2-9} 
\multicolumn{1}{|c|}{\multicolumn{1}{c}{\textit{Terme}}} & \multicolumn{1}{c|}{Fréquence} & \multicolumn{1}{c|}{\textsc{TF-IDF}} & \multicolumn{1}{c|}{\textsc{BM25}} & \textsc{BERT} & \multicolumn{1}{c|}{Fréquence} & \multicolumn{1}{c|}{\textsc{TF-IDF}} & \multicolumn{1}{c|}{\textsc{BM25}} & \textsc{BERT} \\ \hline
%Amyotrophie protopathique                              & \multicolumn{1}{r|}{3}         & \multicolumn{1}{r|}{0,12}   & \multicolumn{1}{r|}{0,82} & \multicolumn{1}{r|}{0,03}      & \multicolumn{1}{r|}{0}         & \multicolumn{1}{r|}{0}      & \multicolumn{1}{r|}{0}    & \multicolumn{1}{r|}{0}         \\ \hline
Arthrite déformante                                    & \multicolumn{1}{r|}{30}         & \multicolumn{1}{r|}{0,16}   & \multicolumn{1}{r|}{0,45} & \multicolumn{1}{r|}{0,80}      & \multicolumn{1}{r|}{24}         & \multicolumn{1}{r|}{0,02}   & \multicolumn{1}{r|}{\textbf{0,50}} & \multicolumn{1}{r|}{0,40}      \\ \hline
Ataxie locomotrice                                     & \multicolumn{1}{r|}{559}        & \multicolumn{1}{r|}{0,35}   & \multicolumn{1}{r|}{0,05} & \multicolumn{1}{r|}{0,83}      & \multicolumn{1}{r|}{169}        & \multicolumn{1}{r|}{0,08}   & \multicolumn{1}{r|}{0,25} & \multicolumn{1}{r|}{0,39}      \\ \hline
Atrophie musculaire                                    & \multicolumn{1}{r|}{1105}       & \multicolumn{1}{r|}{0,20}   & \multicolumn{1}{r|}{0,02} & \multicolumn{1}{r|}{0,84}      & \multicolumn{1}{r|}{1465}       & \multicolumn{1}{r|}{0,43}   & \multicolumn{1}{r|}{0,15} & \multicolumn{1}{r|}{0,42}      \\ \hline
Atrophie progressive                                   & \multicolumn{1}{r|}{40}         & \multicolumn{1}{r|}{0,14}   & \multicolumn{1}{r|}{0,27} & \multicolumn{1}{r|}{0,72}      & \multicolumn{1}{r|}{22}         & \multicolumn{1}{r|}{0,02}   & \multicolumn{1}{r|}{\textbf{0,53}} & \multicolumn{1}{r|}{0,39}      \\ \hline
Catalepsie                                             & \multicolumn{1}{r|}{681}        & \multicolumn{1}{r|}{0,54}   & \multicolumn{1}{r|}{0,07} & \multicolumn{1}{r|}{0,88}      & \multicolumn{1}{r|}{975}        & \multicolumn{1}{r|}{0,28}   & \multicolumn{1}{r|}{0,15} & \multicolumn{1}{r|}{0,39}      \\ \hline
%Cirrhose de muscle                                     & \multicolumn{1}{r|}{1}         & \multicolumn{1}{r|}{0,05}   & \multicolumn{1}{r|}{0,82} & \multicolumn{1}{r|}{0,02}      & \multicolumn{1}{r|}{0}         & \multicolumn{1}{r|}{0}      & \multicolumn{1}{r|}{0}    & \multicolumn{1}{r|}{0}         \\ \hline
Épilepsie                                              & \multicolumn{1}{r|}{414}       & \multicolumn{1}{r|}{0,09}   & \multicolumn{1}{r|}{0,02} & \multicolumn{1}{r|}{0,78}      & \multicolumn{1}{r|}{577}       & \multicolumn{1}{r|}{0,12}   & \multicolumn{1}{r|}{0,10} & \multicolumn{1}{r|}{0,41}      \\ \hline
Hystérie                                               & \multicolumn{1}{r|}{5775}       & \multicolumn{1}{r|}{0,51}   & \multicolumn{1}{r|}{0,01} & \multicolumn{1}{r|}{0,74}      & \multicolumn{1}{r|}{4934}       & \multicolumn{1}{r|}{0,45}   & \multicolumn{1}{r|}{0,05} & \multicolumn{1}{r|}{0,41}      \\ \hline
Langue                                                 & \multicolumn{1}{r|}{2695}        & \multicolumn{1}{r|}{0,24}   & \multicolumn{1}{r|}{0,01} & \multicolumn{1}{r|}{0,72}     & \multicolumn{1}{r|}{3591}       & \multicolumn{1}{r|}{0,11}   & \multicolumn{1}{r|}{0,02} & \multicolumn{1}{r|}{0,41}      \\ \hline
Maladie de Parkinson                                   & \multicolumn{1}{r|}{75}         & \multicolumn{1}{r|}{0,21}   & \multicolumn{1}{r|}{0,23} & \multicolumn{1}{r|}{0,81}     & \multicolumn{1}{r|}{130}        & \multicolumn{1}{r|}{0,09}   & \multicolumn{1}{r|}{0,35} & \multicolumn{1}{r|}{0,37}      \\ \hline
Paralysie bulbaire                                     & \multicolumn{1}{r|}{149}         & \multicolumn{1}{r|}{0,27}   & \multicolumn{1}{r|}{0,15} & \multicolumn{1}{r|}{0,89}     & \multicolumn{1}{r|}{93}         & \multicolumn{1}{r|}{0,09}   & \multicolumn{1}{r|}{0,52} & \multicolumn{1}{r|}{0,40}      \\ \hline
Paralysie rhumatismale                                 & \multicolumn{1}{r|}{8}         & \multicolumn{1}{r|}{0,07}      & \multicolumn{1}{r|}{0,67}    & \multicolumn{1}{r|}{0,86}         & \multicolumn{1}{r|}{14}         & \multicolumn{1}{r|}{0,02}   & \multicolumn{1}{r|}{\textbf{0,68}} & \multicolumn{1}{r|}{0,44}      \\ \hline
Sclérose latérale                                      & \multicolumn{1}{r|}{445}        & \multicolumn{1}{r|}{0,30}   & \multicolumn{1}{r|}{0,06} & \multicolumn{1}{r|}{0,88}      & \multicolumn{1}{r|}{127}         & \multicolumn{1}{r|}{0,09}   & \multicolumn{1}{r|}{0,37} & \multicolumn{1}{r|}{0,41}      \\ \hline
Sclérose en plaque disséminées                         & \multicolumn{1}{r|}{45}         & \multicolumn{1}{r|}{0,25}   & \multicolumn{1}{r|}{0,47} & \multicolumn{1}{r|}{0,87}      & \multicolumn{1}{r|}{12}         & \multicolumn{1}{r|}{0,02}   & \multicolumn{1}{r|}{\textbf{0,83}}    & \multicolumn{1}{r|}{0,40}      \\ \hline
Somnambulisme                                          & \multicolumn{1}{r|}{847}        & \multicolumn{1}{r|}{0,49}   & \multicolumn{1}{r|}{0,05} & \multicolumn{1}{r|}{0,89}      & \multicolumn{1}{r|}{3410}       & \multicolumn{1}{r|}{\textbf{1}}   & \multicolumn{1}{r|}{0,15} & \multicolumn{1}{r|}{0,43}      \\ \hline
\end{tabular}
\caption{Calcul de pertinence des concepts selon les métriques \textsc{TF-IDF}, \textsc{BM25} et \textsc{BERT} dans les corpus \og{}Charcot\fg{} et \og{}Autres\fg{}.}
\end{table}
\label{tab:calculs_stat}
\end{landscape}


